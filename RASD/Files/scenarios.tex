\subsection{Scenarios}
\subsubsection{Scenario 1}
Mario is the CEO of a manufacturing company in Milan, his company collaborates with many shops located around in the region. Once a month he must visit each shop to get a report of administrative and management activities. To organize and take into account the travel time for the appointments, he uses Travlendar+. After having downloaded the application and registered his data, Mario has the possibility to create a meeting into the app for every appointment that he has in his personal calendar. During the meeting creation, he needs to specify the following information: location for the meeting, starting and ending time, name and type of the appointment. The app automatically calculates if the time between the end of the first meeting and the start of the second one is enough for the journey; in case the time is too short and the second meeting wouldn't be reachable in time, a warning is shown.

\subsubsection{Scenario 2}
Luca is a young architect of Milan, and in addition to using the latest graphic system to realize his project, he usually produces a demonstration model for his customers scattered around the city. The great capabilities of Luca permits to him to have a lot of appointments, so he uses Travlendar+ to manage the events. Luca has also set the preferences to use car sharing systems for work appointments, in this way he can carry his models around the town without risking of breaking them. The preferences system implemented in the application shows  to the user the best itinerary that respects settings and user's preferences.

\subsubsection{Scenario 3}
Travlendar+ is the perfect application for Giovanni, father of family and financial advisor.
With the application Giovanni can easy manage his travel time to reach every type of appointment. Travlendar+ always suggests to him the perfect way to reach a meeting, in case of work meetings the primary suggestion is public transportation, instead in case of family meetings the primary suggestion is his own car. When Giovanni chooses the public transportations for his movement he can also buy the ticket directly on the app, in a simple way he can select the payment system and in a few tap he receives the ticket on his email. Furthermore in case of strike the app sends a notification alert to the user and it will recalculate the travel based on the new options.

\subsubsection{Scenario 4}
Antonio is an event organizer and meantime he is a food’s lover; this is the reason why in his calendar is always presents a lunch in one of the best restaurant of Milan. Antonio is a daily user of Travlendar+ for this reason, the app in fact force a mandatory break of at least 30 minutes for lunch. In this way Antonio can has his lunch break in a period of time between 11.30 am and 2.30 pm and he can can arrive in time for the next appointment. Over the lunch break, Antonio usually plans a break in the afternoon, with Travlendar+ he can do this and the minimum time for the break is 5 minutes

\subsubsection{Scenario 5}
Alex is a olympic champion of foot race, when he is not training he is often invited to attend some sport conferences. Alex always suggests that it is really important to do physical activity not only during the training sessions but also when we move around the city. For this reason Alex can not do to use Travlendar+, indeed he chooses the bike for his movements and also selects the option to minimize the carbon footprint. Furthermore the application notifies Alex when the weather conditions are not optimal and suggests him to use another travel mean to reach his desiderated location.
