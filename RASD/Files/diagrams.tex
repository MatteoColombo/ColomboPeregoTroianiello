\subsection{Functional Requirements}

\subsubsection{Use Case Diagram}
\begin{figure}[!h]
\centering
\includegraphics[scale=0.8]{images/UseCaseDiagram}
\caption{Use Case Diagram}
\label{ref:usecasediagram}
\end{figure}

\clearpage
\subsubsection{Use Case Descriptions}

In this section are listed some common or significant use cases derivable from
the Use Case diagram.

%Visitor's registration%
\begin{center}
\begin{longtable}{p{5cm} p{8cm}}
\multicolumn{2}{ c }{\textbf{Visitor's registration}} \\ \hline 
\endfirsthead
\endhead
\multicolumn{2}{ c }{{Continued on next page}} \\ \hline
\endfoot
\hline
\caption{Visitor's registration}
\label{ref:visitorsregistration}
\endlastfoot
\textbf{Actors:} & Visitor \\ 
\textbf{Goals:} & G.1 \\ 
\textbf{Input conditions:} & The visitor is on the home page. \\
\textbf{Event flow:} &  \begin{enumerate}
						\item
						The visitor clicks any button and is redirected to the login page.
						\item
						The visitor clicks on the “Sign Up” button to start the registration process.
						\item
						The visitor fills all the mandatory fields.
						\item
						The visitor clicks on the “Sign Up“ button.
						\item
						The system send the data to the server.
						\item
						The system sends a confirmation email to the new user.
						\end{enumerate}\\ 
\textbf{Output conditions:} & The visitor successfully ends the registration process and is redirected to the login page. From now he can login to the application and start using that. \\ 
\textbf{Exception:} & \begin{enumerate}
						\item
						The visitor is already an user.
						\item
						The visitor inserts not valid information in one or more mandatory fields.
						\item
						The visitor chooses an email that is associated with another user.
						\item
						The server is unreachable.
					\end{enumerate}
All exceptions are handled notifying the issue to the visitor and taking back the Event Flow to the point 2. \\
\end{longtable}
\end{center}

%User's login%
\begin{center}
\begin{longtable}{p{5cm} p{8cm}}
\multicolumn{2}{ c }{\textbf{User's login}} \\ \hline 
\endfirsthead
\endhead
\multicolumn{2}{ c }{{Continued on next page}} \\ \hline
\endfoot
\hline
\caption{User's login}
\label{ref:userslogin}
\endlastfoot
\textbf{Actors:} & User \\ 
\textbf{Goals:} & G.1 \\ 
\textbf{Input conditions:} & The user is on the login page. \\
\textbf{Event flow:} & \begin{enumerate}
				\item
				The user inserts his credentials into the “email” and “password” fields.
				\item
				The user clicks on the “Login” button in order to access.
			\end{enumerate} \\ 
\textbf{Output conditions:} & The user is successfully redirected to the
home page.\\ 
\textbf{Exception:} & \begin{enumerate}
				\item
				The user inserts a not valid email.
				\item
				The user inserts a not valid password.
				\item
				The server is unreachable.
			\end{enumerate}
All exceptions are handled notifying the issue to the user and taking back the Event Flow to the point 1. \\
\end{longtable}
\end{center}

%User creates a meeting%
\begin{center}
\begin{longtable}{p{5cm} p{8cm}}
\multicolumn{2}{ c }{\textbf{Create a meeting}} \\ \hline 
\endfirsthead
\endhead
\multicolumn{2}{ c }{{Continued on next page}} \\ \hline
\endfoot
\hline
\caption{Create a meeting}
\label{ref:createameeting}
\endlastfoot
\textbf{Actors:} & User \\ 
\textbf{Goals:} & G.2, G.3 \\ 
\textbf{Input conditions:} & The user is already logged into the system and is into the schedule page. \\
\textbf{Event flow:} & \begin{enumerate}
				\item
				The user clicks on the “New meeting” button to start the creation process.
				\item
				The user inserts the information  into the fields and the type.
				\item
				The user clicks on the “Create meeting” button.
			\end{enumerate}\\ 
\textbf{Output conditions:} & The user is successfully redirected to the
meeting page.\\ 
\textbf{Exception:} & \begin{enumerate}
				\item
				The name isn’t valid.
				\item
				The start hour doesn’t precede the end hour.
				\item
				The start hour and end hour precede the current date.
				\item
				Exists an another meeting with the same hours. 
				\item
				Travel mean is unavailable. 
			\end{enumerate}
These exceptions are handled notifying the issue to the user and taking back the Event Flow to the point 2.
If the meeting is unreachable, this use case is extended by "Resolve the problem of unreachable meeting".
\\
\end{longtable}
\end{center}

%Resolve the problem of unreachable meeting%
\begin{center}
\begin{longtable}{p{5cm} p{8cm}}
\multicolumn{2}{ c }{\textbf{Resolve the problem of unreachable meeting}} \\ \hline 
\endfirsthead
\endhead
\multicolumn{2}{ c }{{Continued on next page}} \\ \hline
\endfoot
\hline
\caption{Resolve the problem of unreachable meeting}
\label{ref:resolvetheproblemofunreachablemeeting}
\endlastfoot
\textbf{Actor:} & User \\ 
\textbf{Goals:} & G.2, G.3 \\ 
\textbf{Input conditions:} & The user is creating a meeting and this is unreachable.\\
\textbf{Event flow:} & The applicaiton shows a warning message and the user can either click "Edit" to change the meeting information or click "Create" and continue creating the meeting even if it is unreachable in time.\\ 
\textbf{Output conditions:} & The user is successfully redirected to the
meeting page.\\ 
\textbf{Exception:} & All exceptions are the same as those the use case "Create a meeting". \\
\hline
\end{longtable}
\end{center}

%User travels to the meeting%
\begin{center}
\begin{longtable}{p{5cm} p{8cm}}
\multicolumn{2}{ c }{\textbf{Travel to the meeting}} \\ \hline 
\endfirsthead
\endhead
\multicolumn{2}{ c }{{Continued on next page}} \\ \hline
\endfoot
\hline
\caption{Travel to the meeting}
\label{ref:traveltothemeeting}
\endlastfoot

\textbf{Actors:} & User \\ 
\textbf{Goals:} & G.4, G.8 \\ 
\textbf{Input conditions:} & The user is already logged into the system and is into the schedule page. \\
\textbf{Event flow:} & \begin{enumerate}
				\item
				The user clicks on the desired meeting and is redirected into the meeting information.
				\item
				The user clicks on “Navigate” and goes into navigate page.
				\item
				The navigate page helps the user with indications.
				\item
				When the user arrives at the desired location, the application notifies him.
			\end{enumerate} \\ 
\textbf{Output conditions:} & The user is successfully redirected to the
schedule.\\ 
\textbf{Exception:} & \begin{enumerate}
				\item
				The GPS is unavailable.
				\item
				Internet connection isn't working.				
				\item
				The location isn’t found. 
			\end{enumerate}
All exceptions are handled notifying the issue to the user and taking back the Event Flow to the point 1. \\
\hline
\end{longtable}
\end{center}

%User selects preferences and filters options%
\begin{center}
\begin{longtable}{p{5cm} p{8cm}}
\multicolumn{2}{ c }{\textbf{Setup preferences}} \\ \hline 
\endfirsthead
\endhead
\multicolumn{2}{ c }{{Continued on next page}} \\ \hline
\endfoot
\hline
\caption{Setup preferences}
\label{ref:setuppreferences}
\endlastfoot

\textbf{Actors:} & User \\ 
\textbf{Goals:} & G.5, G.7 \\ 
\textbf{Input conditions:} & The user is already logged into the system and is into the home page. \\
\textbf{Event flow:} & \begin{enumerate}
				\item
				The user clicks on “Setting” button and is redirected into setting page.
				\item
				The user inserts his preferences about travel means and lunch timetable.
				\item
				The user confirms the new information with “Confirm” button.
			\end{enumerate}\\ 
\textbf{Output conditions:} & The user stays in the setting page.\\ 
\textbf{Exception:} & \begin{enumerate}
				\item
				The user doesn’t insert numbers in the lunch fields.
			\end{enumerate}
All exceptions are handled notifying the issue to the user and taking back the Event Flow to the point 2. \\
\hline
\end{longtable}
\end{center}

\clearpage
\subsubsection{Sequence Diagrams}

\begin{figure}[!h]
\centering
\includegraphics[scale=0.6]{images/VisitorRegistrationDiagram}
\caption{Visitor Registration Diagram}
\label{ref:visitorregistrationdiagram}
\end{figure}

\clearpage
\begin{figure}[!h]
\centering
\includegraphics[scale=0.5]{images/UserLoginDiagram}
\caption{User Login Diagram}
\label{ref:userlogindiagram}
\end{figure}

\clearpage
\begin{figure}[!h]
\centering
\includegraphics[scale=0.6]{images/CreateMeetingDiagram}
\caption{Create Meeting Diagram}
\label{ref:createmeetingdiagram}
\end{figure}

\clearpage
\begin{figure}[!h]
\centering
\includegraphics[scale=0.7]{images/TravelToTheMeetingDiagram}
\caption{Travel To The Meeting Diagram}
\label{ref:traveltothemeetingdiagram}
\end{figure}

\clearpage
\begin{figure}[!h]
\centering
\includegraphics[scale=0.6]{images/SetupPreferencesDiagram}
\caption{Setup Preferences Diagram}
\label{ref:setupprefencesdiagram}
\end{figure}
