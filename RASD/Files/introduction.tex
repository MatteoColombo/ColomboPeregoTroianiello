\subsection{Purpose}
This document is the baseline for project planning and for software evaluation of Travlendar+; it describes the system in terms of functional and nonfunctional requirements, analysing the needs of the customer in order to model the system.

Travlendar+ is a calendar-based application for people who have problems with scheduling working and personal appointments at various locations all across Milan. The application aims at simplifying the life of people by automatically computing travels and by organizing the daily schedules.

\subsection{Scope}
Travlendar+ is a calendar-based application whose purpose is to help people to easily schedule working or personal meetings around Milan.

The system-to-be will allow users to plan travels and meetings without worrying of being late or to miss lunch and, thanks to its high customizability, people will be able to save a lot of time. Alerts will be given when meetings are created at unreachable locations and travel means will be computed depending on day hours and weather.

\subsubsection{Current System}
Travlendar+ will rely on many online services provided by other companies.\\
An existing payment system will be implemented, as well as a maps and weather service that will be used to compute the best travel means.

The journeys will use public transports, trains and bike sharing systems that are already present in the city.

\newcounter{gcount1}
\subsubsection{Goals}
\begin{list}
{\bfseries{}[G.\arabic{gcount1}]~}
{
\usecounter{gcount1}
}
\item
Allow the user to create meetings.
\item
Show warnings in case of unreachable appointment.
\item
Suggest travel means depending on the appointment and the day.
\item
Allow the user to select preferences and to filter options.
\item
Schedule the user’s lunch break.
\item
Allow the user to schedule breaks during the day.
\item
Assist the user during the travel

\end{list}