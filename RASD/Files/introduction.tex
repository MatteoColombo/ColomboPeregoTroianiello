\subsection{Purpose}
This document is the baseline for project planning and for software evaluation of Travlendar+; it describes the system in terms of functional and nonfunctional requirements, analysing the needs of the customer in order to model the system.\\
Travlendar+ is a calendar-based application for people who have problems with scheduling working and personal appointments at various locations all across Milan. The application aims at simplifying the life of people by automatically computing travels and by organizing the daily schedules.

\subsection{Scope}
Travlendar+ is a calendar-based application whose purpose is to help people to easily schedule working or personal meetings around Milan.
Travlendar+ is intended to all people that need to organize your daily events and movements. The system-to-be will allow users to plan travels and meetings without worrying of being late or to miss lunch and, thanks to its high customizability, people will be able to save a lot of time. Alerts will be given when meetings are created at unreachable locations and travel means will be computed depending on day hours and weather.

\subsection{Definitions, Acronyms, Abbreviations}

\subsubsection{Definitions}
\renewcommand{\labelitemi}{$-$}
\begin{itemize}
\item
Schedule: set of meetings of the same day.
\item
Calendar: set of the schedules and from this you can select a specific schedule.
\item
Meeting: event in the schedule of the user that has a location and a start and end hour.
\item
Break: event that has a custom time range in which it can be schedule, with a minummum duration of 5 minutes.
\item
Lunch: it's a particular break with no customizable range time and minimum duration of 30 minutes.
\item
Travel pass: represents all type of passes (daily, weekly, monthly and yearly pass) for public transportations.
\item
Level: pollution level of a travel travel mean. 
\item
Event type: the type of a meeting (e.g. family, personal, work, etc). 
\end{itemize}


\subsubsection{Acronyms}
\renewcommand{\labelitemi}{$-$}
\begin{itemize}
\item
RASD: Requirement Analysis and Specification Document.
\item
IEEE: Institute of Electrical and Electronic Engineers.
\item
API: Application Programming Interface.
\end{itemize}

\subsubsection{Abbreviations}
\renewcommand{\labelitemi}{$-$}
\begin{itemize}
\item
$[$G.x$]$: the goal number x.
\item
$[$R.x.y$]$: the requirement number y of the goal x.
\item
$[$D.x$]$: the domain assumption number x.
\end{itemize}

\subsection{Reference Documents}
\renewcommand{\labelitemi}{$-$}
\begin{itemize}
\item
Specification Document: “Mandatory Project Assignments.pdf”.
\item
\href{http://ieeexplore.ieee.org/servlet/opac?punumber=6146377}{\color{Black}{IEEE Std 29148-2011 - ISO/IEC/IEEE International Standard - Systems and software engineering.}}
\item
Alloy Specification:\href{http://alloy.mit.edu/alloy/}{\color{Black}{"Software Abstractation", Daniel Jackson.}}
\item
Alloy:\href{http://tmancini.di.uniroma1.it/teaching/courses/2007-2008/mfis/materiale/progetti/Pagliaro\%20-\%20Alloy.relazione.pdf}{\color{Black}{"Alloy e l'Analyzer versione 4.0", "Alessandro Pagliaro"}}

\end{itemize}

\subsection{Document Structure}
This document is essentialy structured in four part:
\begin{itemize}
\item
Section 1: Introduction, this part gives a little description of the project and the definitions about terms and acronyms used in the document.
\item
Section 2: Overall Description, gives more information about the software product in particular its functions, constraints and assumptions.
\item
Section 3: Specific Requirements, describes the complete requirements of the project. You can see a list of possible scenarios, the use cases and the sequence diagram of the specific actions.
\item
Section 4: Alloy Description, this part is the representation of the structure of the project in Alloy language. You can see the entire code and the representation of one generated world.
\item
Section 5: This section contains the details about the efforts of each member of the group and the information about the software used in the writing of the document.
\end{itemize}