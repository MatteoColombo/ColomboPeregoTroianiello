\subsection{Product Perspective}
Travlendar+ will be developed from scratch and it will be a mobile application that will require the user to have a smartphone with an internet connection and an integrated GPS system.

The application will allow the user to digitalize his schedule and will easily manage it.
When the user creates a new meeting, he is asked to provide the coordinates of the location, so that the application can compute the travel time to reach the meeting.
In this way the application makes sure that the user isn't ever late, because it permits to create a meeting only if it is reachable in time.

The application proposes only the best route for each travel that complies with the user preferences; this is achieved by using some external services like maps, public transportation and weather forecast providers.

During the journey, the application will assist the users by showing them the correct route or noticing them when they have to get off from public services.
Furthermore, through an external payment system, users are able to buy travel passes for their travels and visualize them directly in the application.


\subsection{Product Functions}

\newcounter{gcount}
\newcounter{rcount}
\begin{list}
{\bfseries{}[G.\arabic{gcount}]~}
{
\usecounter{gcount}
}
	\item Users can create an account.
    \item Allow the users to create meetings.
	\begin{list}
	{\bfseries{}[R.\arabic{gcount}.\arabic{rcount}]~}
	{
	\usecounter{rcount}
	}
        \item To create a meeting, the user must be logged in the application with an active account.
        \item Each meeting must have a location, which can be either an address or coordinates.
        \item Each meeting must have a starting and ending hour.
        \item Each meeting must have a name.
        \item Each meeting must have a type.
    \end{list}
    \item Show warnings in case of unreachable appointments.
    \begin{list}
	{\bfseries{}[R.\arabic{gcount}.\arabic{rcount}]~}
	{
	\usecounter{rcount}
	}
        \item A warning is given when scheduling a meeting, if the location is unreachable with the available travel means.
        \item In case of strikes or exceptional events (e.g. train failures), a warning is given to the user. 
        \item Users can create meetings even if a warning is shown.
    \end{list}
    \item Suggest travel means depending on the appointment and the day.
    \begin{list}
	{\bfseries{}[R.\arabic{gcount}.\arabic{rcount}]~}
	{
	\usecounter{rcount}
	}
        \item Weather conditions must be taken in account when proposing a travel system.
        \item Different kind of appointments will be reached with different travel means.
    \end{list}
    \item Allow the users to select preferences and to filter options.
    \begin{list}
	{\bfseries{}[R.\arabic{gcount}.\arabic{rcount}]~}
	{
	\usecounter{rcount}
	}
        \item The application must support many travel means and transport systems.
		\item Travel means may be subject to constraints (e.g. user can walk for at most 500m).
		\item Application must show results consistent with the user’s preferences.
		\item Users should be allowed to specify their own passes.
		\item Users should be able to specify their home address.
    \end{list}
    \item Schedule the users’ lunch break.
    \begin{list}
	{\bfseries{}[R.\arabic{gcount}.\arabic{rcount}]~}
	{
	\usecounter{rcount}
	}
        \item The minimum lunch break duration must be 30 minutes. 
		\item Users must be able to select a time slot in which lunch should be scheduled.

    \end{list}
    \item Allow the users to schedule breaks during the day.
    \begin{list}
	{\bfseries{}[R.\arabic{gcount}.\arabic{rcount}]~}
	{
	\usecounter{rcount}
	}
        \item The minimum break duration must be 5 minutes.
    \end{list}
    \item Assist the users during the travel.
    \begin{list}
	{\bfseries{}[R.\arabic{gcount}.\arabic{rcount}]~}
	{
	\usecounter{rcount}
	}
	\item Users should be able to buy transportation tickets.
	\item The system must locate the nearest vehicle of the selected sharing system. 
	\end{list}
\end{list}

\subsection{User Characteristics}

\subsubsection{Principal Actors}

\renewcommand{\labelitemi}{$-$}
\begin{itemize}
\item
Visitor: a person using Travlendar+ without being signed-up. He is able to proceed with registration or log-in.
\item
User: a person has successful login and can use the app services. He can manage his preferences and his appointments.
\end{itemize}

\subsubsection{Secondary Actors}
There are some secondary actors such as third party service providers, that are needed by the system to retrieve information, used to perform payments or to compute the travel options. 


\subsection{Constraints}
\subsubsection{Hardware Constraints}
\begin{enumerate}
\item
To use the application, the user must have a smartphone; either Android or iOS.\\
For Android the minimum supported OS version is Android 5.0 Lollipop.\\
For iOS the minimum required OS version is iOS 8.
\item
The application requires an Internet connection.
\item
The application requires that devices have an integrated GPS system.
\end{enumerate}

\subsubsection{Software Constraints}
\begin{enumerate}
\item
If the users decide to use the bike sharing systems, they must install the systems applications on their phones.
\end{enumerate}

\subsubsection{Safety Constraints}
\begin{enumerate}
\item
The system must guarantee that users data are stored in a safety way and that are used only within the application.
\end{enumerate}

\subsubsection{Regulatory Constraints}
\begin{enumerate}
\item
The system must ask for the users permission to acquire, store and elaborate their location.
\item
The system must complain with the local laws.
\item
It is responsibility of the user to complain with the rules and the local laws.
\end{enumerate}

\subsection{Assumptions and Dependencies}
\subsubsection{Domain Assumptions}
\newcounter{tcount}
\begin{list}
{\bfseries{}[D.\arabic{tcount}]~}
{
\usecounter{tcount}
}
\item
The time slot in which lunch can be scheduled is between 11.30am and 2.30pm.
\item
The minimum duration for a lunch break is 30 minutes.
\item
Breaks work in the same way of lunch, but their time slot can be customized by the user.
\item
The minimum duration of a break is 5 minutes.
\item
A meeting is always associated to a position.
\item
When a user adds a new meeting, if the application detects that it causes one of the following errors, e can either choose to force the creation of the meeting or edit it and change the parameters.
Errors:
\begin{enumerate}
\item
The meeting overlaps with another.
\item
The meeting is unreachable.
\item
The meeting makes another unreachable.
\item
The meeting doesn't leave enough time for lunch.
\end{enumerate}

\item
If the user decides to force the creation of the meeting even if the application returns a warning, it is his responsibility to reach the location in time.
\item
In case the user decides to force the creation of a meeting that doesn’t leave enough time for lunch, the time for the lunch break is not reserved.
\item
In case of a long travel that overlaps with the lunch break and doesn’t give enough time to the lunch in (but there is enough time after the lunch slot), it may be splitted in two parts, separated by the break. The same applies for afternoon/morning breaks.
\item
The application gives higher priority at moving breaks at the beginning of their dedicated time slot.
\item
The application tries to maximize the duration of the breaks.
\item
The daily schedule starts and ends always at the user’s house.
\item
An event starts and ends always in the same calendar day.
\item
GPS has a precision of 3m.
\item
When creating a meeting, the position typed by the user is used as meeting location to compute the travel time; while during the navigation or when the user is attending a meeting, the application retrieves his real location through GPS and updates travels times with it.
\item
The user can select a preferred travel mean for each type of meeting, so that depending on the type, the application prefers different travel means. For example, if for family meetings “car” is the preferred travel mean, if it is available, it would be the suggested mean, even if better options exist.
\item
Travel means can be subjected to different type of constraints, some chosen by the user other by the application.
\item
Travel means are classified by their footprint level.Travel passes can be of different types:
\begin{enumerate}
\item
Single trip.
\item
Daily pass.
\item
Weekly pass.
\item
Monthly pass.
\item
Yearly pass.
\end{enumerate}

\item
A travel pass can be used only for one transportation mean. This means taht a bus ticket can't be used for a tram journey.
\item
Single trip tickets are not named and can be bought without owning an account on the website of the service provider.
\item
Cars and Bike are of two types: shared and owned.
\item
At least one vehicle of a sharing system can always be found within a 10 minute walk.
\item
Each vehicle of a sharing system can be located through GPS.
\item
During a single journey, many different travel means can be used.
\item
Public transports are always on time.
\item
Strikes or public transportation fault are communicated on the website of the service provider. 
\item
The system suggests to the user the best journey that complies with the users preferences.
\item
Meetings can be of three types: family, work and personal.
\item
The server is used for account related purposes and as backup.
\item
When a new meeting is created, the application synchronizes the schedule with the server.
\item
If when a new meeting is created the connection to the server isn't available, the meeting is created anyhow and the synchronization will be done when the connection returns.
\item
If a conflict is generated when the user tries to change the time slot dedicated to a break, a warning is shown and the change fordbidden.
\end{list}