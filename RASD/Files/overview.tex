\subsection{Product Perspective}
Travlendar+ will be developed from scratch and it will be a mobile application that will require the user to have a smartphone with an internet connection and an integrated GPS system.

The application will allow the user to digitalize his schedule and will easily manage it.
When the user creates a new meeting, he is asked to provide the coordinates of the location, so that the application can compute the travel time to reach the meeting.
In this way the application makes sure that the user isn't ever late, because it permits to create a meeting only if it is reachable in time.

The application proposes only the best route for each travel that complies with the user preferences; this is achieved by using some external services like maps, public transportation and weather forecast providers.

During the journey, the application will assist the users by showing them the correct route or noticing them when they have to get off from public services.
Furthermore, through an external payment system, users are able to buy travel passes for their travels and visualize them directly in the application.

\subsubsection{Class Diagram}
The classes Event, Public Transportation, Travel Mean, Owned Vehicle and Shared Vehicle are abstract.\\
The classes that implement/extend Constraint, Level, Owned Mean, Shared Mean and Public Transportation are omitted for clarity.

\begin{figure}[h]
\centering
\includegraphics[scale=0.45]{images/classdiagram}
\caption{Class Diagram}
\label{fig:classdiagram}
\end{figure}


\begin{sidewaysfigure}
    \centering
	\includegraphics[scale=0.6]{images/classdiagram}
	\caption{Class Diagram}
	\label{fig:classdiagramrotate}
\end{sidewaysfigure}

\clearpage

\subsection{Product Functions}

\newcounter{gcount}
\newcounter{rcount}
\begin{list}
{\bfseries{}[G.\arabic{gcount}]~}
{
\usecounter{gcount}
}

\item Users can create new meetings.
\begin{list}
	{\bfseries{}[R.\arabic{gcount}.\arabic{rcount}]~}
	{
	\usecounter{rcount}
	}
\item A meeting requires a location, which is either coordinates or an address.
\item A meeting requires a starting and ending hour.
\item A meeting requires a name.
\item A meeting requires a type.
\item A meeting may have a description.
\end{list}
\item Warnings are shown in case a new meeting is unreachable or causes conflicts.
\begin{list}
	{\bfseries{}[R.\arabic{gcount}.\arabic{rcount}]~}
	{
	\usecounter{rcount}
	}
\item A warning should be given at the creation of a meeting if it is unreachable or if it is causes conflicts.
\item A warning should be given in case of exceptional events that make one or more meetings unreachable.
\item A warning forbids the user to force the creation of a meeting.
\end{list}
\item Travel means are suggested depending on different constraints.
\begin{list}
	{\bfseries{}[R.\arabic{gcount}.\arabic{rcount}]~}
	{
	\usecounter{rcount}
	}
\item Different travel means should be suggested depending on the weather conditions.
\item Different type of meetings should be reached with different/specific travel means.
\item Different travel means should be suggested depending on the hour of the day.
\item Travel means can have constraints on the length of their journey section.
\end{list}
\item Users can set their preferences and select between different options.
\begin{list}
	{\bfseries{}[R.\arabic{gcount}.\arabic{rcount}]~}
	{
	\usecounter{rcount}
	}
\item The application must support different travel means and transportation systems.
\item Users can select constraints on travel means.
\item The application must show results consistent with the users preferences.
\item Users should be able to specify the travel passes they own.
\item Users must specify their home address.
	\item Users can disable travel means.
	\item Users can enable the option to minimize air pollution.
    \end{list}
\item Lunch breaks are automatically scheduled by the application.
\begin{list}
	{\bfseries{}[R.\arabic{gcount}.\arabic{rcount}]~}
	{
	\usecounter{rcount}
	}
\item The minimum length of a lunch break is 30 minutes.
\item The lunch break must be scheduled between 11.30 am and 2.30pm.
\end{list}
\item Users can add custom breaks that are automatically scheduled.
\begin{list}
	{\bfseries{}[R.\arabic{gcount}.\arabic{rcount}]~}
	{
	\usecounter{rcount}
	}
\item The minimum length of a lunch break is 5 minutes.
\item Users can set as many breaks as they want.
\item Users can select the time slot in which they want their break to be scheduled.
\end{list}
\item Users are assisted during their journey.
\begin{list}
	{\bfseries{}[R.\arabic{gcount}.\arabic{rcount}]~}
	{
	\usecounter{rcount}
	}
\item The application should notify the user when he has to get off from a transportation system.
\item The applications should give directions to the users when they are driving, cycling or walking.
\item Users can buy travel passes through the application.
\item The application must locate the nearest vehicle of the selected sharing system.
\end{list}
\end{list}

General requirements:
\begin{list}
	{\bfseries{}[R.G.\arabic{rcount}]~}
	{
	\usecounter{rcount}
	}
\item The user must be logged in the application to perform all the actions.
\item An account requires an username.
\item An account requires an email address.
\item An account requires a password.
\item The email address must be unique.
\end{list}

\subsection{User Characteristics}

\subsubsection{Principal Actors}

\renewcommand{\labelitemi}{$-$}
\begin{itemize}
\item
Visitor: a person using Travlendar+ without being signed-up. He is able to proceed with registration or log-in.
\item
User: a person has successful login and can use the app services. He can manage his preferences and his appointments.
\end{itemize}

\subsubsection{Secondary Actors}
There are some secondary actors such as third party service providers, that are needed by the system to retrieve information, used to perform payments or to compute the travel options. 

\subsection{Assumptions and Dependencies}
\subsubsection{Domain Assumptions}
\newcounter{tcount}
\begin{list}
{\bfseries{}[D.\arabic{tcount}]~}
{
\usecounter{tcount}
}
\item
The time slot in which lunch can be scheduled is between 11.30am and 2.30pm.
\item
The minimum duration for a lunch break is 30 minutes.
\item
Breaks work in the same way of lunch, but their time slot can be customized by the user.
\item
The minimum duration of a break is 5 minutes.
\item
A meeting is always associated to a position.
\item
When a user adds a new meeting, if the application detects that it causes one of the following errors, it forces the user to edit the information of the meeting. Errors:

\begin{enumerate}
\item
The meeting overlaps with another.
\item
The meeting is unreachable.
\item
The meeting makes another unreachable.
\item
The meeting doesn't leave enough time for lunch.
\end{enumerate}


\item
In case of a long travel that overlaps with the lunch break and doesn’t give enough time to the lunch in (but there is enough time after the lunch slot), it may be splitted in two parts, separated by the break. The same applies for afternoon/morning breaks.
\item
The application gives higher priority at moving breaks at the beginning of their dedicated time slot.
\item
The application tries to maximize the duration of the breaks.
\item
The daily schedule starts at the user's home and ends at the last meeting of the day.
\item
An event starts and ends always in the same calendar day.
\item
GPS has a precision of 3m.
\item
When creating a meeting, the position typed by the user is used as meeting location to compute the travel time; while during the navigation, the application retrieves his real location through GPS and updates travels times with it.
\item
The user can select a preferred travel mean for each type of meeting, so that depending on the type, the application prefers different travel means. For example, if for family meetings “car” is the preferred travel mean, if it is available, it would be the suggested mean, even if better options exist.
\item
Travel means can be subjected to different type of constraints, some chosen by the user other by the application.
\item
Travel means are classified by their footprint level.Travel passes can be of different types:
\begin{enumerate}
\item
Daily pass.
\item
Weekly pass.
\item
Monthly pass.
\item
Yearly pass.
\end{enumerate}

\item
A travel pass can be used only for one transportation mean. For example, a bus ticket can't be used for a tram journey.
\item
Cars and Bikes are of two types: shared and owned.
\item
At least one vehicle of a sharing system can always be found within a 10 minute walk.
\item
Each vehicle of a sharing system can be located through GPS.
\item
During a single journey, many different travel means can be used.
\item
Public transports are always on time.
\item
Strikes or public transportation fault are communicated on the website of the service provider. 
\item
The system suggests to the user the best journey that complies with the users preferences.
\item
Meetings can be of three types: family, work and personal.
\item
The server is used for account related purposes and as backup.
\item
When a new meeting is created, the application synchronizes the schedule with the server.
\item
If when a new meeting is created the connection to the server isn't available, the meeting is created anyhow and the synchronization will be done when the connection returns.
\item
If a conflict is generated when the user tries to change the time slot dedicated to a break, a warning is shown and the change fordbidden.
\end{list}