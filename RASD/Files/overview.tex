\subsection{Product Perspective}
Travlendar+ will be developed from scratch and it will be a mobile application that will require the user to have a smartphone with an internet connection.

The application will be able to buy transportations ticket in a secure and traceable way by using a trusted payment system. The application also interacts with a partner that give to the user weather informations used to calculate the best route to reach the location of the events, in this way depending on the weather conditions different type of transportations are suggested.

 The user can also modify the preferences for your trip in such a way to  receive suggested of transportation in base of the selected preferences.

\subsection{Product Functions}

\newcounter{gcount}
\newcounter{rcount}
\begin{list}
{\bfseries{}[G.\arabic{gcount}]~}
{
\usecounter{gcount}
}
    \item Allow the users to create meetings.
	\begin{list}
	{\bfseries{}[R.\arabic{gcount}.\arabic{rcount}]~}
	{
	\usecounter{rcount}
	}
        \item To create a meeting, the user must be logged in the application with an active account.
        \item The users must be able to create an account, so that they can utilise all the features of the application..
        \item Each meeting must have a location, which can be either an address or coordinates.
        \item Each meeting must have a starting and ending hour.
        \item Each meeting must have a name.
        \item Each meeting must have a type.
    \end{list}
    \item Show warnings in case of unreachable appointments.
    \begin{list}
	{\bfseries{}[R.\arabic{gcount}.\arabic{rcount}]~}
	{
	\usecounter{rcount}
	}
        \item A warning is given when scheduling a meeting, if the location is unreachable with the available travel means.
        \item In case of strikes or exceptional events (e.g. train failures), a warning is given to the user. 
        \item Users can create meetings even if a warning is shown.
    \end{list}
    \item Suggest travel means depending on the appointment and the day.
    \begin{list}
	{\bfseries{}[R.\arabic{gcount}.\arabic{rcount}]~}
	{
	\usecounter{rcount}
	}
        \item Weather conditions must be taken in account when proposing a travel system.
        \item Different kind of appointments will be reached with different travel means.
    \end{list}
    \item Allow the users to select preferences and to filter options.
    \begin{list}
	{\bfseries{}[R.\arabic{gcount}.\arabic{rcount}]~}
	{
	\usecounter{rcount}
	}
        \item The application must support many travel means and transport systems.
		\item Travel means may be subject to constraints (e.g. user can walk for at most 500m).
		\item Application must show results consistent with the user’s preferences.
		\item Users should be allowed to specify their own passes.
		\item Users should be able to specify their home address.
    \end{list}
    \item Schedule the users’ lunch break.
    \begin{list}
	{\bfseries{}[R.\arabic{gcount}.\arabic{rcount}]~}
	{
	\usecounter{rcount}
	}
        \item The minimum lunch break duration must be 30 minutes. 
		\item Users must be able to select a time slot in which lunch should be scheduled.

    \end{list}
    \item Allow the users to schedule breaks during the day.
    \begin{list}
	{\bfseries{}[R.\arabic{gcount}.\arabic{rcount}]~}
	{
	\usecounter{rcount}
	}
        \item The minimum break duration must be 5 minutes.
    \end{list}
    \item Assist the users during the travel.
    \begin{list}
	{\bfseries{}[R.\arabic{gcount}.\arabic{rcount}]~}
	{
	\usecounter{rcount}
	}
	\item Users should be able to buy transportation tickets.
	\item The system must locate the nearest vehicle of the selected sharing system. 
	\end{list}
\end{list}

\subsection{User Characteristics}

\subsubsection{Principal Actors}

\renewcommand{\labelitemi}{$-$}
\begin{itemize}
\item
Visitor: a person using Travlendar+ without being signed-up. He is able to proceed with registration or log-in.
\item
User: a person has successful login and can use the app services. He can manage his preferences and his appointments.
\end{itemize}

\subsubsection{Secondary Actors}
There are some secondary actors such as third party service providers, that are needed by the system to retrieve information, used to perform payments or to compute the travel options. 


\subsection{Constraints}
\subsubsection{Hardware Constraints}
\begin{enumerate}
\item
To use the application, the user must have a smartphone; either Android or iOS.\\
For Android the minimum supported OS version is Android 5.0 Lollipop.\\
For iOS the minimum required OS version is iOS 8.
\item
The application requires an Internet connection.
\item
The application requires that devices have an integrated GPS system.
\end{enumerate}

\subsubsection{Software Constraints}
\begin{enumerate}
\item
If the users decide to use the bike sharing systems, they must install the systems applications on their phones.
\end{enumerate}

\subsubsection{Safety Constraints}
\begin{enumerate}
\item
The system must guarantee that users data are stored in a safety way and that are used only within the application:
\end{enumerate}

\subsubsection{Regulatory Constraints}
\begin{enumerate}
\item
The system must ask for the users permissione to acquire, store and elaborate their location.
\item
The system must complain with the local laws.
\end{enumerate}

\subsection{Assumptions and Dependencies}
\subsubsection{Domain Assumptions}
\begin{enumerate}
\item
Tickets are not named and they can be bought without owning an account on the service provider's website.
\item
During a travel between two meetings, combinations of different travel means could be used.
\item
Bus, Trains and all the other public services are always on schedule.
\item
An available vehicle of a sharing system service can always be located within a 10 minutes walk.
\item
Every vehicle of a sharing system service can be located through GPS.
\item
Strikes and system malfunctions are always reported on the companies websites.
\item
An account is composed by: username, email and password.
\item
Email address is unique for each account.
\item
Public transport passes are composed by: validity period, public transport to which they are related and validity area or route.
\item
Mobile devices used by the users have GPS and can be located.
\item
If "home location" is not set, the position retrieved through GPS is used as initial location for the daily travels.
\item
System suggests onlye the best journey option to the users.
\item
Meetings can be of three types: personal, family and working.
	
\end{enumerate}