\subsection{Overview: High-level components and their interaction}
This chapter describes the system components both at the physical and logical level.
The main high level components of the system are the following:
\begin{itemize}
\item
\textbf{Database:} The database server which is responsible for the data storage and retrieval. It doesn’t implement any logic as it is used only for data storing purposes. This layer must guarantee that the ACID properties are respected.
\item
\textbf{Application server:}  The application server contains all the logic on the server side of the system. This layer implements RESTful APIs and is used for registrations, login, backup and restore purposes.
\item
\textbf{Mobile Application:} The application consists in the client side of the application. It is installed on the users’ devices and implements most of the logic of the system. 
For the account/backup purposes it communicates directly with the application server, while for all the other functions is standalone.
\end{itemize}
The components are structured in a three layer application, shown in the following figure.
\begin{figure}[!h]
\centering
\includegraphics[scale=0.8]{images/highlevelstructure}
\caption{High Level Structure}
\label{ref:highlevelstructure}
\end{figure}

\clearpage
\subsection{Component View}
\subsubsection{High-Level Component View}
\begin{figure}[!h]
\centering
\includegraphics[scale=0.4]{images/ComponentDiagramSystem}
\caption{High Level Component View}
\label{ref:highlevelcomponentview}
\end{figure}
Component to be developed:
\begin{itemize}
\item
\textbf{Application:} It is the core of the system, it manages all information provided by the others services and performs the majority of the functions. It provides the client access to the entire system.
\item
\textbf{Server:} This component has a backup role. It receives the data from the application and provides them when necessary (e.g. during the login operation).
\end{itemize}
Component to be integrated in the system:
\begin{itemize}
\item
\textbf{MapsSystem:} It is the provider of the maps and all necessary information for the computation of the journey.
\item
\textbf{SharedCarSystem, SharedBikeSystem:} Those component provide all information (availability, location and costs) respectively shared cars and shared bikes.
\item
\textbf{PublicTransportSystem:} It provides all information about the public transport of the city.
\item
\textbf{WeatherSystem:} It provides the weather forecasts in the city.
\item
\textbf{PaymentSystem:} This component provides the payment service.
\end{itemize}

\clearpage
\subsubsection{Application System}
\begin{figure}[!h]
\centering
\includegraphics[scale=0.35]{images/ComponentDiagramApplicationSystem}
\caption{Application Component Diagram}
\label{ref:componentdiagramapplicationsystem}
\end{figure}
\begin{itemize}
\item
\textbf{NavigationController:} The component that provides navigation utilities using the Maps APIs and GPS.
\item
\textbf{GPSManager:} The component that handles and gives the GPS information.
\item
\textbf{SharedController:} The component that provides information of all shared systems.
\item
\textbf{PaymentHandler:} The component that handles the payment operations to buy a ticket for a public transport. It ensures that the user is able to successfully complete the payment.
PublicTransportController: The component manages the availability information and the timetable of all public transport.
\item
\textbf{DataManager:} The component that implements and provides through an appropriate interface the methods for accessing the data of our system and it takes care to send the data to the server.. This intermediate component between the entities of the the model and the other components will facilitate extendibility.
\item
\textbf{Model:} It represent how the data are structured (specified in distinct diagram) in the application and ready to be stored by the server on the database.
\item
\textbf{LoginManager:} The component that handles the login operation.
\item
\textbf{Configurator:} The component that offers the configuration functionalities
to customize a set of parameters of the user account.
\item
\textbf{Event Manager:} The component that handles the operations to create and manage an event.
\item
\textbf{Journey Manager:} It manages the user journey.
\item
\textbf{View Controller:} The component that handles the update of the GUI and the retrieval of the user input through the interface.
\item
\textbf{GUI:} Implementation of the presentation layer of the application.
\end{itemize}

\subsubsection{Server System}
\begin{figure}[!h]
\centering
\includegraphics[scale=0.4]{images/ComponentDiagramServer}
\caption{Server Component Diagram}
\label{ref:componentdiagramserver}
\end{figure}
\begin{itemize}
\item
\textbf{RequestDispatcher:} It handles the requests from the application.
\item
\textbf{DataAccessManager:} The component that manages access to the database using a specific driver.
\item
\textbf{DBMS:} The system that will take care of the management of the data, integrated in our system using a specific driver.
\end{itemize}


\clearpage
\subsection{Deployment View}
This diagram purpose is to show the hardware components of our system and where the code is going to run.
\begin{figure}[!h]
\centering
\includegraphics[scale=0.4]{images/DeploymentDiagram}
\caption{Deployment Diagram}
\label{ref:deploymentdiagram}
\end{figure}

\clearpage
\subsection{Runtime View}

\clearpage
\subsection{Component Interfaces}

\clearpage
\subsection{Selected Architectural styles and patterns}
The following architectural styles have been used:
\begin{itemize}
\item
\textbf{Model-Control-View:} It is used for the main components of the system. It’s a really good choice of design that allows to keep clear the role of every component and that makes the system easier to deploy and maintain.
\item
\textbf{Client-Server:} This pattern is a good practice for a distributed system. It is used between the application server (client) queries the DB (server) and the application (client) communicates with the application server.
\item
\textbf{Service-oriented Architecture:} It is used by the system for the communication between the server and the user’s device (RESTful). The SOA allows to think at a higher level of abstraction, by looking at the component interfaces and not at their specific implementation. SOA style also improves modularity: by making service description, discovery and binding explicit, it is easier to build new plugins and test single modules independently.
\item
\textbf{Fat Client:} The fat client paradigm is implemented because the interaction between user’s device and the server hasn’t a central role in the system behavior . Having a fat client in our case is an advantage because most application logic is on the user’s device, which has sufficient computing power and is able to manage concurrency issue efficiently.
\end{itemize}
