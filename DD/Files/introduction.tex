\subsection{Purpose}
The purpose of this document is to describe in details the concepts expressed in the RASD file, and to specify how the system will be implemented and which logical and physical components will be used.

\subsection{Scope}
Travlendar+ is a calendar-based application whose purpose is to help people to easily schedule working or personal meetings around Milan.
Given what was written in the RASD, the main issues on which we will focus one are the following:
\begin{itemize}
\renewcommand\labelitemi{-}
\item
Provide the services,
\item
Allow the user to access the system and all its functions,
\item
Provide a reliable system that can help the users in their daily life.
\end{itemize}

To solve these issue, the development should focus on:
\begin{itemize}
\renewcommand\labelitemi{-}
\item
Extensibility
\item
Accessibility
\item
Reliability
\end{itemize}

For these reasons the system will be developed on a multi-tier architecture that will have to dialogue with almost standalone clients.
In fact, thanks to the fact that most of the system logic is implemented on the client, the application will be more reliable and more efficient.


\subsection{Definitions, Acronyms and Abbreviations}

\subsubsection{Definitions}
\begin{itemize}
\renewcommand\labelitemi{-}
\item
Schedule: set of meetings of the same day.
\item
Calendar: set of the schedules and from this you can select a specific schedule.
\item
Meeting: event in the schedule of the user that has a location and a start and end hour.
\item
Break: event that has a custom time range in which it can be schedule, with a minummum duration of 5 minutes.
\item
Lunch: it's a particular break with no customizable range time and minimum duration of 30 minutes.
\item
Travel pass: represents all type of passes (daily, weekly, monthly and yearly pass) for public transportations.
\item
Level: pollution level of a travel travel mean. 
\item
Event type: the type of a meeting (e.g. family, personal, work, etc). 
\end{itemize}


\subsubsection{Acronyms}
\begin{itemize}
\renewcommand\labelitemi{-}
\item
RASD: Requirement Analysis and Specification Document.
\item
DD: Design Document
\item
IEEE: Institute of Electrical and Electronic Engineers.
\item
API: Application Programming Interface.
\item
JEE: Java Enterprise Edition.
\item
REST: REpresentational State Transfer.
\item
JAX-RS: Java API for RESTful Web Services.
\item
JPA: Java Persistent API.
\item
SQL: Structured Query Language.
\item
DBMS: DataBase Management System.
\item
JSON: JavaScript Object Notation.
\item
O/R: Object/Relational.
\item
UX: User eXperience.
\item
HTTPS: HyperText Transfer Protocol over Secure socket layer.
\end{itemize}

\subsubsection{Abbreviations}
\begin{itemize}
\renewcommand{\labelitemi}{$-$}
\item
$[$G.x$]$: the goal number x.
\item
$[$R.x.y$]$: the requirement number y of the goal x.
\item
$[$R.G.y$]$: the general requirement number y.
\item
$[$D.x$]$: the domain assumption number x.
\end{itemize}

\subsection{Reference Documents}
\begin{itemize}
\renewcommand\labelitemi{-}
\item
Specification Document: “Mandatory Project Assignments.pdf”.
\item
Requirement Analysis and Specification Document: “RASD.pdf”.
\item
\href{http://ieeexplore.ieee.org/document/5167255/}{\color{Black}{IEEE 1016-2009 - IEEE Standard for Information Technology--Systems Design--Software Design Descriptions.}}
\item
\href{https://www.uml-diagrams.org/}{\color{Black}{UML guide site.}}
\end{itemize}

\clearpage
\subsection{Revision History}
\begin{center}
\begin{longtable}{|p{2cm} | p{3cm}| p{8cm}|}
\hline \multicolumn{1}{|c|}{\textbf{Version}} & \multicolumn{1}{c|}{\textbf{Date}} & \multicolumn{1}{c|}{\textbf{Changes}} \\ \hline 
\endfirsthead
\hline
\endhead
\hline \multicolumn{3}{c}{{Continued on next page}} \\
\endfoot
\hline
\caption{Revision History}
\label{ref:revision}
\endlastfoot
v.1.0 & 26/11/2017 & Initial release \\
\hline 
v.1.1 & 27/12/2017 & Update the requirements part in according to the RASD \\  
\end{longtable}
\end{center}

\subsection{Document Structure}
This document is structured in four parts:
\begin{itemize}
\item
Part 1: the first part focuses on the design and it is composed by three parts: Architectural, Algorithm and User Interface Design.\\
In the first part the document focuses on the physical and logical components, on how they are aggregated and on their functions.\\
In the algorithm part, the document analyzes four of the main algorithms of the system.\\
In the User Interface part, the document gives a more precise description of the UI components and explains how they are connected together.
\item
Part 2: in the second part the document focuses on the requirements of the system and specifies how they have been implemented and how they are connected to the components of the system.
\item
Part 3: in the third part the document focuses on the implementation, integration and test plan, explaining how the development team should proceed to test and implement the all the components.
\item
Part 4: in the fourth part the document focuses on the efforts spent to the writing of this document.
\end{itemize}